\documentclass[../main.tex]{subfiles}

\setcounter{section}{18}

\begin{document}

\begin{definition}[Associates, Irreducibles, Primes]
  Let $a \ b \in D$ where $D$ is an integral domain,
  $a$ and $b$ are called associates if $a = ub$ where $u$ is a unit of $D$.
  If $a$ is non-zero and not a unit, and whenever $a = bc$ for some $b \ c \in D$
  implies $b$ or $c$ is a unit, then $a$ is called irreducible.
  If $a$ is non-zero and not a unit, and $a \mid bc$ implies $a \mid b$ or $a \mid c$,
  then $a$ is called a prime.
\end{definition}

\begin{theorem}
  \label{Theorem:18.1}
  In an integral domain, every prime is an irreducible.
\end{theorem}
\begin{proof}
  Let $D$ an integral domain and $p \in D$ a prime. Suppose $p = ab$ for some 
  $a \ b \in D$, then $p \mid ab$ since $p = 1 ab$, which implies $p \mid a$
  or $p \mid b$, we may suppose $p \mid a$.
  Then $a = pc$ for some $c \in D$, therefore $p = pcb$. By cancellation
  (since we are in an integral domain) we know $1 = cb$, therefore $b$ is a unit
  and $\inv{b} = c$.
\end{proof}

\begin{theorem}
  In a principal ideal domain, an element is irreducible iff it is a prime.
\end{theorem}
\begin{proof}
  Since a principal ideal domain is an integral domain, $(\Leftarrow)$ is trivial.

  $(\Rightarrow)$ Let $P$ a principal ideal domain and $p \in P$, 
  and suppose $I = \cyc{q}$ an ideal
  (we know it has form $\cyc{q}$ since we are in a principal ideal domain)
  such that $\cyc{p} \subseteq \cyc{q}$.
  Since $p \in \cyc{q}$, we know $p = qr$ for some $r \in P$,
  according to $p$ is irreducible, we know either $q$ or $r$ is unit.
  If $q$ is unit, then $\cyc{q} = P$.
  If $r$ is unit, then $q = p\inv{r}$, therefore $q \in \cyc{p}$ and $\cyc{p} = \cyc{q}$.
  This shows that $\cyc{p}$ is a maximal ideal, therefore it is a prime ideal,
  and $\cyc{p}$ is prime ideal implies $p$ is prime.
\end{proof}

\begin{lemma}
  \label{Lemma:18.1}
  In a principal ideal domain, any strictly increasing chain of ideals
  $I_0 \subset I_1 \subset \dots$ must be finite.
\end{lemma}
\begin{proof}
  This proof comes from textbook.

  Let $D$ be that principal ideal domain and 
  $I = I_0 \cup I_1 \cup \dots$, 
  for any element $i \in I$ and $a \in D$, $i$ must be an element of an ideal
  in the chain, say $I_i$, then $ia \in I_i \subseteq I$.
  Therefore $I$ is an ideal.

  then $I = \cyc{a}$, and $a$ must belongs
  to an ideal in the chain, say $I_n$. For any ideal $I_m$ in the chain,
  we have $I_m \subseteq I_n$ since $a$ divides the "generator" of $I_m$.
  Therefore $I_n$ is the last member of the chain.
\end{proof}

\begin{definition}[Unique Factorization Domain]
  An integral domain $D$ is a unique factorization domain if:
  \begin{enumerate}
    \item Every non-zero element of $D$ that is not a unit can be written
          as a product of irreducibles of $D$.
    \item The factorization into irreducibles is unique
          up to \textbf{associates and the order of the factors}.
  \end{enumerate}
\end{definition}

Therefore, the factorization $4 = 2 \times 2 = (-2) \times (-2)$ is considered "the same", cause
$2$ is associated with $-2$ by $2 = (-1) 2$ (or $\cyc{2} = \cyc{-2}$).

\begin{theorem}[PID implies UFD]
  Every pincipal ideal domain is a unique factorization domain.
\end{theorem}
\begin{proof}
  Let $D$ a principal ideal domain and $a \in D$ a non-zero, non-unit element.

  We first show that for any reducible, there is an irreducible divides it.
  Suppose $a \in D$ is reducible, then $a = bc$ where $b \ c \in D$ are 
  non-zero and non-unit. If $b$ or $c$ is irreducible, then trivial.
  If both $b$ and $c$ are reducible, then $\cyc{a} \subset \cyc{b}$,
  it is trivial that $\cyc{a} \subseteq \cyc{b}$, if $b \in \cyc{a}$, then
  $b = ad$, which means $a = bc \rightarrow a = adc$, which implies $c$ is a unit
  but it isn't. Then we perform this algorithm on $b$, and we have a strictly
  increasing chain of ideals $\cyc{a} \subset \cyc{b} \subset \dots$.
  According to the Lemma \ref{Lemma:18.1}, we know the chain is finite,
  therefore the algorithm must stop at some point.
  We found that the algorithm only stops when it meets an irreducible,
  therefore there is a irreducible divides $a$.

  We repeat applying this algorithm to the factor of $a$,
  the factor of the factor of $a$ and so on, then we have a series $a = p_0 p_1 p_2 \dots$
  where $p_i$ are irreducibles. If this series is infinite,
  then $b = p_1 p_2 \cdots$ and $\cyc{a} \subset \cyc{b}$ (properly containing by $p_0$ is not a unit).
  Therefore, we have a infinite strictly increasing chain $\cyc{p_0 p_1 p_2 \dots} \subset \cyc{p_1 p_2 p_3 \dots} \subset \dots$
  which is impossible. Therefore $a = \join[]{p}{n}$ where $p_i$ are irreducibles.

  If $a = \join[]{p}{n} = \join[]{q}{m}$, we induction on $n$:
  \begin{itemize}
    \item Base: If $a = p_0 = \join[]{q}{m}$, then $p_0$ must divides some $q_i$.
          Since $q_i$ is irreducible, we know $p_0$ and $q_i$ are associates.
          Then the product of the remaining irreducibles are a unit by cancellation,
          which only makes sense when there is no remaining irreducibles.
          Therefore $0 = m$.
    \item Induction: If $a = \join[]{p}{n} = \join[]{q}{m}$, then $p_0$
          must divides some $q_i$ and they are associates, that is, $cp_0 = q_i$, 
          we may suppose $i = 1$ since an integral domain is commutative, 
          then by cancellation,
          we have $(c p_1) p_2 \cdots p_n = q_1 q_2 \cdots p_m$. By induction hypothesis,
          we know $n = m$, and they are unique up to associates.
  \end{itemize}
\end{proof}

\begin{definition}[Euclidean Domain]
  An integral domain $D$ is called a Euclidean domain if there is a function
  $d : D^* \rightarrow \N$ (called measure) where $D^*$ is $D$ without $0$,
  such that:
  \begin{enumerate}
    \item $d(a) \leq d(ab)$ for all non-zero $a \ b \in D$.
    \item For any $a \ b \in D$ and $b \neq 0$, then $a$ can be written in the
          form of $a = bq + r$ where $q \ r \in D$ and 
          either $r = 0$ or $d(r) < d(b)$.
  \end{enumerate}
\end{definition}

\begin{example}
  The ring $\Z$ is an Euclidean domain with $d(a) = |a|$.
\end{example}

\begin{example}
  For any field $F$, $\poly{F}$ is an Euclidean domain with $d = \deg$
\end{example}

\begin{theorem}
  Every Euclidean domain is a principal integral domain.
\end{theorem}
\begin{proof}
  Let $D$ an Euclidean domain and $I$ a non-zero ideal of $D$.
  Let $b \in I$ with $d(b)$ is minimal, we claim $I = \cyc{b}$.
  For any element $a$ in $I$, $a$ can be written in the form of $a = bq + r$,
  if $r = 0$, then $a \in \cyc{b}$; if $r \neq 0$, then $d(r) < d(b)$,
  since $r \in I$ (cause $a \ b \in I$ and $I$ is an ideal), it contradicts the
  assumption that $d(b)$ is minimal.
\end{proof}

\end{document}