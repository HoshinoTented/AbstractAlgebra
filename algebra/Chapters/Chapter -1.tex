\documentclass[14pt]{extarticle}
\usepackage[T1]{fontenc}
\usepackage[margin=1in]{geometry}
\usepackage{amsthm,amsmath}
\usepackage{hyperref}

\newtheorem{theorem}{Theorem}[section]
\newtheorem{lemma}{Lemma}[section]
\newtheorem{exercise}{Exercise}[section]
\newtheorem*{example}{Example}
\newtheorem{definition}{Definition}[section]
\setcounter{section}{-1}

\DeclareMathOperator{\Ker}{Ker}
\DeclareMathOperator{\Aut}{Aut}
\DeclareMathOperator{\orb}{orb}
\DeclareMathOperator{\stab}{stab}

\newcommand{\inv}[1]{#1^{-1}}
\newcommand{\normalin}{\triangleleft}
\newcommand{\1}{\{e\}}

\begin{document}
This chapter involves group action, but the knowleage is from untrusted sources
\footnote[0]{\url{https://zhuanlan.zhihu.com/p/165163924}}
\footnote[1]{\url{https://www.bananaspace.org/wiki/\%E7\%BE\%A4\%E4\%BD\%9C\%E7\%94\%A8}}.

\begin{definition}[Group Action]
  For any group $G$ and any set $X$.
  A \textbf{group action} of $G$ on $X$ is a mapping $-\cdot- : G \times X \rightarrow X$
  such that:
  \begin{itemize}
    \item Preserve Identity: $\forall x \in X, e \cdot x = x$
    \item Associativity: $\forall g \ h \in G, x \in X, (gh) \cdot x = g \cdot (h \cdot x)$
  \end{itemize}

  We say $G$ is left action on $X$, and the set $X$ is called $G$-set.
\end{definition}

\begin{definition}[Orbit]
  For any group $G$, and a set $X$ where $X$ is a $G$-set.
  For any $x \in X$, the orbit of $x$:
  \[
    \orb(x) = \{ g \cdot x \ | \ g \in G \}
  \]

  In other words, the orbit of $x$ is all the possible 'positions' that $x$ can be sent to.
\end{definition}

\setcounter{section}{7}
\setcounter{exercise}{66}
\begin{exercise}[Orbit is Partition]
  Show that the intersection of distinct orbits is empty set.
\end{exercise}
\begin{proof}
  Suppose $G$ a group and $X$ a $G$-set, $x \ y \in X$.
  If there is a $a \in X$ such that $a \in \orb(x) \cap \orb(y)$,
  then exists $g \in G$ such that $g \cdot x = a$
  and exists $g^\prime \in G$ such that $g^\prime \cdot y = a$.
  Then $(\inv{(g^\prime)} g) \cdot x = y$, therefore $\orb(y) \subseteq \orb(x)$.
  Similarly $\orb(x) \subseteq \orb(y)$.
\end{proof}
\setcounter{section}{-1}

\begin{definition}[Stablizer]
  For any group $G$, and $X$ a $G$-set.
  For any $x \in X$, the stablizer of $x$:
  \[
    \stab(x) = \{ g \in G \ | \ g \cdot x = x \}
  \]
\end{definition}

\begin{lemma}[Stablizer is Subgroup]
  For any group $G$, and $X$ a $G$-set. For any $x \in X$, $\stab(x)$ is a subgroup of $G$.
\end{lemma}
\begin{proof}
  By two-steps:
  \begin{enumerate}
    \setcounter{enumi}{-1}
    \item $\stab(x)$ is not empty since $e \cdot x = x$
    \item For any $g \ h \in \stab(x)$, $(gh) \cdot x = g \cdot (h \cdot x) = g \cdot x = x$.
    \item For any $g \in \stab(x)$, 
      \begin{align*}
        g \cdot x &= x \\
        \inv{g} \cdot (g \cdot x) &= \inv{g} \cdot x \\
        (\inv{g} g) \cdot x &= \inv{g} \cdot x \\
        x &= \inv{g} \cdot x
      \end{align*}
  \end{enumerate}
  Thus $\stab(x)$ is a subgroup of $G$.
\end{proof}

\begin{lemma}
  For any group $G$, and $X$ a $G$-set. For any $x \in X$, $|G| = |\orb(x)||\stab(x)|$
  (or equivalently, $|\orb(x)| = [ G : \stab(x) ]$).
\end{lemma}
\begin{proof}
  Consider the function $\phi(g\stab(x)) = g \cdot x$ 
  from the cosets of $\stab(x)$
  to $\orb(x)$, we will show that it is bijective,
  but first, we need to show that it \textbf{is} a function.
  For any $g\stab(x)$ and $h\stab(x)$, if $g\stab(x) = h\stab(x)$,
  then $g \in h\stab(x)$, which means $g = hs$ where $s \in \stab(x)$.
  Then $\phi(g\stab(x)) = g \cdot x = (hs) \cdot x = h \cdot (s \cdot x) = h \cdot x = \phi(h\stab(x))$.

  \begin{itemize}
    \item One-to-one: For any $g \ h \in G$, $\phi(g\stab(x)) = \phi(h\stab(x))$,
      then $g \cdot x = h \cdot x$, therefore $\inv{g} \cdot (h \cdot x) = x$,
      which implies $(\inv{g}h) \cdot x = x$ and $\inv{g}h \in \stab(x)$.
      Then $g\stab(x) = h\stab(x)$.
    \item Onto: For any $a \in \orb(x)$, then there is $g \in G$ such that $g \cdot x = a$.
          Then $\phi(g\stab(x)) = g \cdot x = a$.
  \end{itemize}
  Thus $\phi$ is bijective and then $|\orb(x)| = [ G : \stab(x) ]$
\end{proof}

\end{document}