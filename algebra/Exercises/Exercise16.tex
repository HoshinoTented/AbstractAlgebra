\documentclass[../main.tex]{subfiles}

\setcounter{section}{16}

\begin{document}

\setcounter{exercise}{12}
\begin{exercise}
  Let $\phi : R \rightarrow S$ a ring homomorphism, 
  define $\overline{\phi} : \poly{R} \rightarrow \poly{S}$
  by $\overline{\phi}(a_nx^n + a_{n - 1}x^{n - 1} + \dots) = \phi(a_n)x^n + \phi(a_{n - 1})x^{n - 1} + \dots$.
  Show that $\overline{\phi}$ is a ring homomorphism.
\end{exercise}

\begin{exercise}
  If $R$ and $S$ are ring isomorphic, then $\poly{R}$ and $\poly{S}$ are ring isomorphic.
\end{exercise}

\setcounter{exercise}{15}
\begin{exercise}
  Let $f(x)$ and $g(x)$ are cubic polynomials with integer coefficients
  such that $f(a) = g(a)$ for four (distinct) integer values $a$.
  Prove that $f(x) = g(x)$, Generalize.
\end{exercise}
\begin{proof}
  Consider $h(x) = f(x) - g(x)$, $\deg h(x) \leq 3$, therefore there are at most $3$
  zeros. However, we found that there are four values $a$ such that $f(a) - g(a) = 0$,
  so $h(x) = 0 \rightarrow f(x) = g(x)$.

  Moreover, we can show that any polynomials with degree $n$ is determined
  by $n + 1$ points.
\end{proof}

\setcounter{exercise}{18}
\begin{exercise}[Degree Rule]
  Let $D$ be an integral domain and $f(x) \ g(x) \in \poly{D}$.
  Prove that $\deg (f(x)g(x)) = \deg f(x) + \deg g(x)$.
\end{exercise}
\begin{proof}
  Let $n = \deg f(x)$ and $m = \deg g(x)$.
  Degree is determined by the leading term, while the leading term of $f(x)g(x)$
  is $f_nx^n g_mx^m = f_ng_m x^{n + m}$. $f_ng_m$ will never be $0$, since $D$ is
  an integral domain.
\end{proof}

\setcounter{exercise}{31}
\begin{exercise}
  Give an example of a polynomial of $\poly{Z_5}$ of positive degree
  that has the property that $f(a) = 1$ for all $a \in Z_5$.
\end{exercise}
\begin{proof}
  Try $(x - 4)(x - 3)(x - 2)(x - 1)x + 1$, normalized $x^5 + 4x + 1$.

  The Path: I was trying to find it directly, but I failed, cause I assume that
  its degree is lower than $5$, which is an unappropriate assumption, because
  $x^5 = x$ is the key of this problem.

  Moreover, consider $f(x) = x^p + (p - 1)x + 1$ for some prime $p$,
  we have $f(a) = 1$ for all $a \in Z_p$.
\end{proof}

% And now, I can solve a exercise which I can't before.

% \setcounter{exercise}{29}
% \begin{exercise}
%   Let $n > 1$ be a non-prime integer. Show that there is a
%   polynomial $f(x) \in \poly{Z_n}$ of degree $n - 1$ that has $n$ distinct zeros.
% \end{exercise}
% \begin{proof}
%   Consider $f(x) = (x - (n - 1))(x - (n - 2))\cdots(x - 2)(x - 1)$.
%   It is easy to see that $\deg f(x) = n - 1$, but we need to show that $f(0) = 0$.
%   We know the constant term of $f(x)$ is $(- (n - 1))(- (n - 2))\cdots(-2)(-1)$,
%   because $n$ is non-prime, there are distinct $a \ b \in Z_n$ such that $a \ b$
% FIXME: a and b may not distinct, consider Z_4 n
% \end{proof}

\setcounter{exercise}{42}
\begin{exercise}
  Let $F$ a field, $f(x)$ and $g(x)$ in $\poly{F}$ and not both zero.
  If there is no polynomial of positive degree in $\poly{F}$ that divides
  both $f(x)$ and $g(x)$, prove that there exist polynomials $h(x)$ and $k(x)$
  in $\poly{F}$ such that $f(x)h(x) + g(x)k(x) = 1$.
\end{exercise}
\begin{proof}
  This problem can be solved by showing $1 \in \cyc{f(x), g(x)}$.
  Consider the ideal $\cyc{f(x), g(x)}$, we know it is principal ideal so that
  there is $h(x) \in \poly{F}$ such that $\cyc{h(x)} = \cyc{f(x), g(x)}$. We
  also know $h(x)$ has the minimum degree in $\cyc{f(x), g(x)}$ and
  there are $s(x) \ t(x) \in \poly{F}$ such that $h(x)s(x) = f(x)$ and $h(x)t(x) = g(x)$,
  therefore $h(x)$ has to have degree $0$.
  So $h(x) = h_0$ and $h_0\inv{h_0} \in \cyc{h(x)}$ since $\cyc{h(x)}$ is an ideal.

  We know every element in $\cyc{f(x), g(x)}$ has form $f(x)h(x) + g(x)k(x)$ for some
  $h(x) \ k(x) \in \poly{F}$ and $1 \in \cyc{f(x) , g(x)}$.
\end{proof}

\begin{exercise}
  Let $F$ a field, $f(x)$ and $g(x)$ in $\poly{F}$ and not both zero.
  A polynomial $d(x) \in \poly{F}$ is said to be a greatest common divisor
  of $f(x)$ and $g(x)$ if $d(x)$ divides both $f(x)$ and $g(x)$,
  and $d(x)$ has maximum degree among all such polynomials.
  Prove that $\cyc{f(x), g(x)} = \cyc{d(x)}$, and there is a unique monic gcd.
\end{exercise}
\begin{proof}
  Trivial.
\end{proof}

\setcounter{exercise}{56}
\begin{exercise}
  For every prime $p$, show that $x^{p - 1} - 1 = (x - 1)(x - 2)\cdots(x - (p - 2))(x - (p - 1))$
  in $\poly{Z_p}$.
\end{exercise}
\begin{proof}
  It is easy to see that both side have degree $p - 1$,
  and for any element $a \in Z_p$, $a$ is a zero of both side, therefore they are equal
  to each other (See Exercise 16.16).
\end{proof}

\begin{exercise}[Wilson's Theorem]
  For every integer $n > 1$, prove that $(n - 1)! = n - 1 \quad (\modu \ n)$
  iff $n$ is prime.
\end{exercise}
\begin{proof}
  ~
  \begin{itemize}
    \item $(\Rightarrow)$
      Suppose $n$ is not prime and does \textbf{NOT} have form $p^2$ where $p$ is prime,
      then there is a pair of zero-divisor that makes the left hand side zero.
      So we suppose $n = p^2$,
      then the product of all element in $U(p^2) \cup \{ p \}$ is $n - 1$,
      then $p = (n - 1)\inv{(\text{product of $U(p^2)$})} \in U(p^2)$.
    \item $(\Leftarrow)$
      By let $x$ in Exercise 16.57 be $0$, we know $-1 = (-1)^{n - 1} (n - 1)!$,
      recall that $-1 = n - 1$ in $Z_n$ (even $n$ is not prime)
      and $a^{n - 1} = 1$ in $U(n)$, since $|U(n)| = n - 1$.
      So $n - 1 = 1 (n - 1)!$.
  \end{itemize}
\end{proof}

\setcounter{exercise}{65}
\begin{exercise}
  Let $R$ a commutative ring with unity, $I$ is a prime ideal of $R$.
  Prove that $\poly{I}$ is a prime ideal of $\poly{R}$.
\end{exercise}
\begin{proof}
  For any $f(x) \ g(x) \in \poly{R}$ where $f(x)g(x) \in \poly{I}$,
  we induction on $(\deg f(x) , \deg g(x))$.
  \begin{itemize}
    \item Base (Left): If $\deg f(x) = 0$, since each coefficients are in $I$,
          we know that either $f_0 \in I$ or $g(x) \in \poly{I}$. If $f(x) = 0$,
          then trivial.
    \item Base (Right):   Ditto.
    \item Induction: Suppose $\deg f(x) = m$ and $\deg g(x) = n$ where $m$ and $n$ are positive.
          Consider the leading coefficient of $f(x)g(x)$, it is produced by $f_m g_n$,
          therefore, one of them is in $I$. We may suppose $f_m \in I$, otherwise
          we just swap them.
          Then $f(x)g(x) = f_mg(x) + f^\prime(x)g(x)$ (where $f^\prime(x)$ is $f(x)$ without leading coefficient),
          we know $f_mg(x) \in \poly{I}$ since $f_m \in I$, then by induction hypothesis,
          we know either $f^\prime(x)$ or $g(x)$ in $\poly{I}$.
          If $f^\prime(x) \in \poly{I}$, so is $f(x) = f_m x^m + f^\prime(x)$,
          otherwise, $g(x) \in \poly{I}$.
  \end{itemize}

  Note that we don't claim which one is in $\poly{I}$ at the beginning, cause we don't have
  sufficient information.
\end{proof}

\setcounter{exercise}{69}
\begin{exercise}
  Let $F$ a field and let $I = \set{f(x) \in \poly{F}}{\forall a \in F, f(a) = 0}$.
  Prove that $I$ is an ideal of $\poly{F}$.
  Prove that $I$ is infinite when $F$ is finite and $I = \0$ when $F$ is infinite.
  Find a monic polynomial $g(x)$ such that $I = \cyc{g(x)}$ when $F$ is finite.
\end{exercise}
\begin{proof}
  $I$ is an ideal cause:
  \begin{itemize}
    \item Non-empty
    \item For any $f(x) \ g(x) \in I$, $a \in F$, $f(a) + g(a) = 0$.
    \item For any $f(x) \in I$, $g(x) \in \poly{F}$, $a \in F$, $f(a)g(a) = 0g(a) = 0$
  \end{itemize}

  Suppose $F$ is finite, then the polynomial $f(x) = (x - a_0)(x - a_1) \cdots$ where $a_i \in F$
  is in $I$, and for any positive integer $n$, $f(x)x^n \in I$ with degree $\deg f(x) + n$,
  therefore $I$ is infinite.
  If $F$ is infinite, then there is no polynomial has infinite zeros except $f(x) = 0$.

  If $F$ is finite, the $f(x)$ above is such monic polynomial.
\end{proof}

\setcounter{exercise}{74}
\begin{exercise}
  Suppose $F$ is a field and there is a ring homomorphism from $Z$ onto $F$.
  Show that $F$ is isomorphic to $Z_p$ for some prime $p$.
\end{exercise}
\begin{proof}
  Why this exercise here...?

  $Z/\Ker \phi$ has to be a integral domain, therefore $\Ker \phi = \cyc{p}$.
\end{proof}

\end{document}