\documentclass[../main.tex]{subfiles}

\setcounter{section}{15}

\begin{document}

\setcounter{exercise}{22}
\begin{exercise}
  Show that the homomorphism preserve idempotent.
\end{exercise}
\begin{proof}
  $\phi(a) = \phi(a^2) = \phi(a)^2$.
\end{proof}

\setcounter{exercise}{35}
\begin{exercise}
  The sum of the squares of three consecutive integers 
  can not be a square.
\end{exercise}
\begin{proof}
  This proof comes from math stackexchange. \par
  For any integer $x$, we found $(x - 1)^2 + x^2 + (x + 1)^2 = 3x^2 + 2$,
  if such square exists, then it must not a multiple of $3$,
  and the remainder should be $2$, therefore, 
  the number we want has form $3n + r$ where $n$ is integer and $0 < r < 3$ 
  (Note that $0 \neq r$ since the number we want is not a multiple of $3$).
  Then $(3n + r)^2 = 9n^2 + 6nr + r^2$, and $1^2 = 1$, $2^2 = 1$.
  Therefore no $r$ such that $r^2 = 2$, so $3x^2 + 2$ can not be a square.
\end{proof}

\setcounter{exercise}{45}
\begin{exercise}
  Prove that any automorphism of a field $F$ is the identity
  from the prime subfield to itself.
\end{exercise}
\begin{proof}
  We know prime subfield is a subfield that does not contain any
  proper non-trivial subfield, therefore it is
  the minimal subfield that contains $1$.
  It is finite if $\chara R \neq 0$ and it is $Q$ if $\chara R = 0$.

  Let $\phi$ a automorphism of $F$, then $\phi(1) = 1$,
  any element in such prime subfield has form $n \cdot \inv{(b \cdot 1)}$
  where $n$ and $b$ are integers. Note that $\phi(n \cdot \inv{(b \cdot 1)})$ is determined by $\phi(1)$,
  and $\phi(1) = 1$, so $\phi$ is the identity.
\end{proof}

\setcounter{exercise}{48}
\begin{exercise}
  Let $R$ and $S$ be commutative rings with unity,
  $\phi$ a homomorphism from $R$ onto $S$ and $\chara R \neq 0$.
  Prove that $\chara S$ divides $\chara R$.
\end{exercise}
\begin{proof}
  Since $\phi$ is onto and $R$ has unity, we know $\phi(1) = 1$.
  Let $\chara R = n$, then $\phi(n \cdot 1) = n \cdot \phi(1) = 0$,
  therefore the order of unity of $S$ under additive divides $n$.
\end{proof}

\setcounter{exercise}{51}
\begin{exercise}
  Show that a homomorphism from a field onto a non-zero ring
  must be an isomorphism.
\end{exercise}
\begin{proof}
  We need to show that such homomorphism $\phi$ is one-to-one.
  Since $F$ a field, we know $\Ker \phi$ is 
  either a zero ideal or $F$ itself. We may suppose $\Ker \phi = F$,
  since another case is trivial.
  Then $\phi(F) = \0$, however, $\phi$ is onto and the codomain is not a zero-ring,
  so $\phi(F)$ cannot be $\0$.
\end{proof}

\begin{exercise}
  Suppose that $R$ and $S$ are commutative ring with unities.
  Let $\phi$ a homomorphism from $R$ to $S$ and let $A$
  be an ideal of $S$:
  \begin{itemize}
    \item If $A$ is prime, show that $\inv{\phi}(A)$ is also prime.
    \item If $A$ is maximal, show that $\inv{\phi}(A)$ is also maximal.
  \end{itemize}
\end{exercise}
\begin{proof}
  If $A$ is prime, for any element $ab \in \inv{\phi}(A)$,
  we have $\phi(ab) \in A$, therefore $\phi(a)$ or $\phi(b)$ in $A$,
  which implies $a$ or $b \in \inv{\phi}(A)$.

  If $A$ is maximal, for any ideal $I$ that properly contains $\inv{\phi(A)}$ in $R$,
  then $\phi(I)$ properly contains $A$ and $\phi(I) = S$,
  therefore $I = \inv{\phi(S)} = R$.
\end{proof}

\begin{exercise}
  Show that the homomorphic image of a principal ideal ring is also a principal ideal ring.
\end{exercise}
\begin{proof}
  Let $\phi$ a homomorphism from a principal ideal ring $R$ onto some ring $S$,
  then $S$ is commutative and has a unity. For any ideal $I$ of $S$,
  $\inv{\phi}(I)$ is a principal ideal, say, $\cyc{r} = rR$,
  then $I = \phi(rR) = \phi(r)\phi(R) = \phi(r)S$, therefore $I$ is a principal ideal ring
  which generated by $\phi(r)$.
\end{proof}

\setcounter{exercise}{56}
\begin{exercise}
  Show that $Z_{mn}$ is ring-isomorphic to $Z_m \oplus Z_n$ when $m$ is coprime to $n$.
\end{exercise}
\begin{proof}
  By Group Theory, we know $Z_{mn}$ is group-isomoprhic to $Z_m \oplus Z_n$,
  then there is an isomorphism $\phi$ that maps $\phi(1)$ to any generator of $Z_m \oplus Z_n$,
  we choose $\phi(1) = (1, 1)$. Then, for any $a \ b \in Z_{mn}$

  \begin{align*}
     & \phi(a \ b) \\
    =& \phi( (a \cdot 1) b) \\
    =& \phi( a \cdot (1b)) \\
    =& a \cdot \phi(b) \\
    =& a \cdot (\phi(1) \phi(b)) \\
    =& (a \cdot \phi(1)) \phi(b) \\
    =& \phi(a \cdot 1) \phi(b) \\
    =& \phi(a) \phi(b)
  \end{align*}
\end{proof}

\begin{exercise}
  Let $m$ and $n$ are distinct positive integer, Show that $mZ \approx nZ$ implies False.
\end{exercise}
\begin{proof}
  Note that a ring isomorphism $\phi : mZ \rightarrow nZ$ 
  is also a (additive) group isomorphism,
  therefore $\phi(m) = n \text{ or } -n$.
  Consider $\phi(m^2)$, we know $n^2 = \phi(m^2) = \phi(m \cdot m)$ since we are in $Z$,
  then $m \cdot \phi(m) = m \cdot (\pm n) = \pm mn$, 
  we get $\pm m = n$ by cancellation (since $Z$ is an integral domain).
  We know both $m$ and $n$ are positive, so $-m = n$ is impossible, therefore $m = n$,
  but we also know $m$ and $n$ are distinct.
\end{proof}

\begin{exercise}
  Let $D$ an integral domain and let $F$ be the field of quotient of $D$.
  For any field $E$ that contains $D$, show that $F$ is ring-isomorphic 
  to some subfield of $E$.
\end{exercise}
\begin{proof}
  Consider the mapping $\phi(a/b) = a\inv{b}$, but we have to show that it \textbf{is} a mapping.
  For any $a/b$ and $c/d$ in $F$ such that $a/b = c/d$, that is, $ad = bc$.
  Then $\phi(a/b) = a\inv{b} = a\inv{b}d\inv{d} = bc\inv{b}\inv{d} = c\inv{d} = \phi(c/d)$
  (recall that $E$ is commutative).

  We claim $\phi$ is a homomorphism from $F$ to $E$, for any $a/b \ c/d \in F$ 
  (We denote $+_F$ as the addition of $F$ and $+$ as the addition of $E$):
  \begin{itemize}
    \item \begin{align*}
       & \phi(a/b +_F c/d) \\
      =& \phi((ad + bc) / bd) \\
      =& (ad + bc)\inv{(bd)} \\
      =& ad\inv{d}\inv{b} + bc\inv{d}\inv{b} \\
      =& a\inv{b} + c\inv{d} \\
      =& \phi(a/b) + \phi(c/d)
    \end{align*}
    \item \begin{align*}
       & \phi(a/b \cdot c/d) \\
      =& \phi(ac/bd) \\
      =& ac\inv{(bd)} \\
      =& a\inv{b}c\inv{d} \\
      =& \phi(a/b)\phi(c/d)
    \end{align*}
  \end{itemize}

  Further more, we hope that $\phi$ is also one-to-one, suppose $\phi(a/b) = \phi(c/d)$,
  we know $a\inv{b} = c\inv{d}$ and then $ad = bc$, which implies $a/b = c/d$.

  Therefore, $F \approx \phi(F)$ where $\phi(F)$ is a subfield of $E$ (it is a field since $F$ is a field).
\end{proof}

\end{document}